\documentclass{article}

% Language setting
% Replace `english' with e.g. `spanish' to change the document language
\usepackage[english]{babel}

% Set page size and margins
% Replace `letterpaper' with `a4paper' for UK/EU standard size
\usepackage[letterpaper,top=2cm,bottom=2cm,left=3cm,right=3cm,marginparwidth=1.75cm]{geometry}

% Useful packages
\usepackage{amsmath}
\usepackage{amssymb}
\usepackage{graphicx}
\usepackage{inconsolata}
\usepackage{minted}
\usepackage[colorlinks=true, allcolors=blue]{hyperref}

\title{Prøve 1 eksamen}
\author{Skrevet av André Hansen}

\begin{document}
\maketitle

\begin{abstract}
Dette er prøve nummer 1
\end{abstract}

\section{Del 2: Med hjelpemidler}

\subsection{Oppgave 1}

\subsubsection{a)}

\begin{figure}[H]
    \centering
    \includegraphics[width=0.7\textwidth]{bilder/opg1.a.1.png}
    \caption{Utregning i cas}
\end{figure}

Jeg henter den deriverte i punktet 0.

Farten er ca $31.21 m/s$

\subsubsection{b)}

Hvordan kan denne løses dersom vi ikke vet hvor målet er?

\subsubsection{c)}

På det høyeste er ballen ca $1.36$ meter over bakken og har en fart på ca $30.58 m/s$

\begin{figure}[H]
    \centering
    \includegraphics[width=0.7\textwidth]{bilder/opg1.b.1.png}
    \caption{Utregning i cas}
\end{figure}

\subsection{Oppgave 2)}

Først utfører jeg en regresjonsanalyse med en enhets oppløsning.

Som regresjonsmodell valgte jeg en sjettegrads polynomfunksjon

\begin{figure}[H]
    \centering
    \includegraphics[width=0.7\textwidth]{bilder/opg2.1.png}
    \caption{Regresjonsanalyse i Geogebra}
\end{figure}

Derreter regner jeg volumet for omdreingslegemet i cas, ved bruke funksjonen fra analysen.

Volumet til pæren er ca $310.72 cm$

\begin{figure}[H]
    \centering
    \includegraphics[width=0.7\textwidth]{bilder/opg2.2.png}
    \caption{Utregning i CAS}
\end{figure}

\subsection{Oppgave 3)}

\subsubsection{a)}

De ulike veridene i modellen $T(x)$ passer veldig bra med opplysningene gitt ovenfor

Ettersom solnedgang varierer i løpet av året refelekterer modellen godt solnedgang i året

\subsubsection{b)}

Tidspunktet på når lyset slår seg på flytter seg med 3 minutter er $46$ og $135$ dager etter nyttår.

\begin{figure}[H]
    \centering
    \includegraphics[width=0.7\textwidth]{bilder/opg3.b.1.png}
    \caption{Utregning i CAS}
\end{figure}

\subsubsection{c)}

Tidspunktet endrer seg raskest for hver $90.9$'ende og $273$'ende dag.

\begin{figure}[H]
    \centering
    \includegraphics[width=0.7\textwidth]{bilder/opg3.c.1.png}
    \caption{Utregning i CAS}
\end{figure}

Da ender den seg med $4$ minutter og $9$ sekunder.

\subsection{Oppgave 4)}

\subsubsection{a)}

Den rekursive sammenhengen for summen av kubikktallene er :

$$S_{n+1} = S_n + (n+1)^3$$

Vet ikke hvordan man finne eksplissit til sum med polynomfunksjon

\subsubsection{b)}

\begin{verbatim}
S = 0

for n in range(1, 51):
    S += n**3

print(f"Summen er:", S)
\end{verbatim}

\subsubsection{c)}

Vet ikke hvordan man finne eksplissit til sum med polynomfunksjon

\subsection{Oppgave 5)}

Vi kan finne avstanden ved : $Avstand fra sentrum til planet - radius$

$r = \frac{|B-A|}{2}=2.45$

$\gamma : x + 2y + 2z = 14$

Sentrum
$S = A + \frac{B-A}{2} = (3, 0, -3)$

Definerer et punkt som er det nærmeste punktet $S$ på $\gamma$

Dermed kan vi finne avstanden fra sirkelen til planet ved

\begin{align*}
    |\vec{SQ}| &= \frac{|1 \cdot 3 + 2 \cdot 0 + 2 \cdot -3 -14|}{\sqrt{1^2 + 2^2 + 2^2}} - r \\ 
    &= -sqrt{6} + 17 / 3 \\
    \approx 3.22
\end{align*}

\begin{figure}[H]
    \centering
    \includegraphics[width=0.7\textwidth]{bilder/opg5.a.1.png}
    \caption{Utregning i CAS}
\end{figure}

\subsubsection{b)}

$\alpha = \gamma$

Denne er parallel med $\gamma$ og har samme avstand.

\subsection{6)}

Det er ikke mulig siden resultatet blir : $e^x \cdot a_1$ som aldri kan bli $0$

\begin{figure}[H]
    \centering
    \includegraphics[width=0.7\textwidth]{bilder/opg6.1.png}
    \caption{Utregning i CAS}
\end{figure}

\bibliographystyle{alpha}
\bibliography{sample}

\end{document}
