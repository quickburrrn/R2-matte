\documentclass{article}

% Language setting
% Replace `english' with e.g. `spanish' to change the document language
\usepackage[english]{babel}

% Set page size and margins
% Replace `letterpaper' with `a4paper' for UK/EU standard size
\usepackage[letterpaper,top=2cm,bottom=2cm,left=3cm,right=3cm,marginparwidth=1.75cm]{geometry}

% Useful packages
\usepackage{amsmath}
\usepackage{amssymb}
\usepackage{graphicx}
\usepackage{inconsolata}
\usepackage{minted}
\usepackage[colorlinks=true, allcolors=blue]{hyperref}

\title{R2 Eksamen Høst 2025}

\begin{document}
\maketitle

\section{Oppgave 1}

\subsection{a)}

Jeg løser denne oppgaven ved å derivere funksjonen for posisjon:

$$|\vec{r}(2)'| = \text{fart etter 2 sekunder}$$

\begin{figure}[H]
    \centering
    \includegraphics[width=0.7\textwidth]{bilder/oppgave1.a.png}
    \caption{Utregning i CAS}
\end{figure}

Farten til ubåten etter 2 sekunder er ca $10.3 m/s$

\subsection{b)}

Jeg løser denne oppgaven ved å finne et globalt minimum for $t\in[0, 60]$

Dette gjør jeg ved å finne nullpunktene til z-aksen til den deriverte.

$$Z(\vec{r}(2)') = 0$$

Dette gir meg tiden ubåten er på dypeste.

Derreter setter jeg tiden på det dypeste i posisjonsformelen og ser på z-aksen

\begin{figure}[H]
    \centering
    \includegraphics[width=0.7\textwidth]{bilder/oppgave1.b.png}
    \caption{Utregning i CAS}
\end{figure}

Ubåten befinner seg $312.5 m$ under havet på det dypeste.

\subsection{c)}

Kan kan finne ut om fiskestimen kolliderer med ubåten ved å se på avstanden mellom ubåten og fiskestimen på det nærmeste.

Jeg velger å lage en model for avstanden mellom ubåten og fistesteamen, jeg kaller denne for $d$ for distance

$\vec{d}(t)=|\vec{r}(t)-\vec{s}(t)|$

Dersom jeg deriverer denne funksjonen kan jeg se hvordan avstand endrer seg over tid.

Dette betyr at detsom jeg løser likningen:

$\vec{d}(t)'=0$

Kan jeg finne tidspunktet for den nærmeste avstanden.

Derreter setter jeg resultate fra likningen i funksjonen for avstand.

Til slutt sammenligner jeg avstanden med størrelsene til ubåten og fiskestimen.

\begin{figure}[H]
    \centering
    \includegraphics[width=0.7\textwidth]{bilder/oppgave1.c.1.png}
    \caption{Utregning i CAS}
\end{figure}

Dermed er vi at den minste avstanden mellom fiskestimen og ubåten er ca $40m$

Fiskestimen og ubåten ikke store nokk for at de skal kolidere.

\section{Oppgave 2)}

\subsection{a)}

For å løse denne oppgaven utførere jeg regresjonsanalyse i Geogebra og velger en egnet modell.

\begin{figure}[H]
    \centering
    \includegraphics[width=0.7\textwidth]{bilder/oppgave2.a.1.png}
    \caption{Legger til datapunkt i Geogebra}
\end{figure}

Jeg valgte $\sin$, det ga best tilnærming fordi punktene danner en harmoniske svigninger.

\begin{figure}[H]
    \centering
    \includegraphics[width=0.7\textwidth]{bilder/oppgave2.a.2.png}
    \caption{regresjonsanalyse}
\end{figure}

Deremd la jeg modellen i grafikkfeltet.

\begin{figure}[H]
    \centering
    \includegraphics[width=0.7\textwidth]{bilder/oppgave2.a.3.png}
    \caption{Moddel for tid(x) og volt(y)}
\end{figure}

\subsection{b)}

Jeg kan finne tidspunktene spenningen er $230 V$ ifølge modellen ved å løse likningen:

$$U(t)=230, \quad t\in(0, 0.02)$$

\begin{figure}[H]
    \centering
    \includegraphics[width=0.7\textwidth]{bilder/oppgave2.b.1.png}
    \caption{Utregning i CAS}
\end{figure}

Dermed ser vi at modellen sier at spenningen er på $230 V$ etter ca $0.00254s$ og $0.00746s$

\subsection{c)}

Jeg starter med å finne $T$

$$T=P=\frac{2\pi}{c}=\frac{2\pi}{314}$$

\begin{figure}[H]
    \centering
    \includegraphics[width=0.7\textwidth]{bilder/oppgave2.c.1.png}
    \caption{Utregning i CAS}
\end{figure}

Målingene er veldig nære men ikke mattematisk riktig. Truls ville nok ha avrundet opp til $230V$, men dette kommer ann på hvor mye nøyaktighet an aksepterer.

\section{Opgpave 3}

\subsection{a)}

Jeg kan løse denne oppgaven med å definere $CCL_4$ i kroppen som en funksjon av dager:

$$CCL_4=f(x)=2 \cdot \sum_{i=0}^{x} (0.82)^i=2 \cdot \frac{0.82^x-1}{0.82 - 1}$$

Dermed kan jeg løse likningen : 

$$f(x)=10$$

for å finne ut av hvor mange netter Sofie må sove på soverommet før konsentrasjonen av $CCI_4$ beggyner å bli farlig.

\begin{figure}[H]
    \centering
    \includegraphics[width=0.7\textwidth]{bilder/oppgave3.a.1.png}
    \caption{Utregning i CAS}
\end{figure}

\subsection{b)}

For å løse denne oppgaven definrer jeg en funksjon for summen av en konvergerende uendelig rekke:

$$f(x)=2 \cdot \lim_{n \rightarrow \infty} \sum_{i=0}^{n} (x)^i = \frac{2}{1-x}, \quad |x|< 1$$

Dermed kan jeg løse likningen:

$$f(x)=10$$

\begin{figure}[H]
    \centering
    \includegraphics[width=0.7\textwidth]{bilder/oppgave3.b.1.png}
    \caption{Utregning i CAS}
\end{figure}

Her ser vi at artikelen antar at en voksen person skiller ut 20 prosent $CCI_4$ hver dag

\section{Opgpave 4}

\subsection{a)}

Grunnen til at den ene metoden vil gi litt for høy verdi og den andre metoden litt for lav verdi er på grun av hvordan høyre og vesnte tilnærming fungerer.

Høyresum vil altid starte rektangelene på høyreside dermed blir summen høyre

Vensteside vil altid starte på venste dermed komme litt under grafen, som betyr at summen blir litt lavere.

Her en en illusrtasjon på hvordan tilnærmingene fungerer:

\begin{figure}[H]
    \centering
    \includegraphics[width=0.7\textwidth]{bilder/oppgave4.a.2.png}
    \caption{Her ser du hvordan rekgangelsummene er unøyaktige}
\end{figure}

\subsection{b)}

Jeg valgte å skrive om funksjonen til en funksjon som regner med midtpunkttilnærming:

$$\int_{a}^{b} f(x) dx \approx \sum_{i = 1}^{n} f(m_i) \cdot \Delta x , \text{ der } m_i = a + (i - \frac{1}{2}) \cdot \Delta x$$

\begin{figure}[H]
    \centering
    \includegraphics[width=0.7\textwidth]{bilder/oppgave4.b.1.png}
    \caption{Resutlatet av å kjøre koden}
\end{figure}

Integralet blir ca $36.4$ ved midtpunkttilnærming med 100 rektangler

\bibliographystyle{alpha}
\bibliography{sample}

\end{document}
