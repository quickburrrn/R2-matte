\documentclass{article}

% Language setting
% Replace `english' with e.g. `spanish' to change the document language
\usepackage[english]{babel}

% Set page size and margins
% Replace `letterpaper' with `a4paper' for UK/EU standard size
\usepackage[letterpaper,top=2cm,bottom=2cm,left=3cm,right=3cm,marginparwidth=1.75cm]{geometry}

% Useful packages
\usepackage{amsmath}
\usepackage{amssymb}
\usepackage{graphicx}
\usepackage[colorlinks=true, allcolors=blue]{hyperref}

\title{Innlevering 1: Del 2-5}
\author{Skrevet av André Hansen}

\begin{document}
\maketitle

\begin{abstract}
Dette er det løsning for første innlevering som dekker del 2-5 i pensum. Jeg kommer til å behandle denne innleverigen på lik linje som selve eksamenen.
\end{abstract}

\section{Uten hjelpemidler}

\subsection{Oppgave 1: Betrakt rekka 5+9+13+17+21+...}

\subsubsection{Hva er det tolvte leddet i denne rekka?}

Først defineres en eksplisitt formel med den eksplisitte formelen for aritmetiske rekker

\begin{align*}
    a_n &= k(n-1)+a_1 \\
    a_1 &= 5 \\
    k &= a_2 - a_1 = 4 \\
    a_n &= 4(n-1)+5
\end{align*}

så brukes resultatet for å regne $a_{12}$

\begin{align*}
    a_{12} &= 4(12-1)+5 = 49
\end{align*}

\subsubsection{Hva er summen av de tolv første leddene}

Først og fremst defineres en eksplisitt formel med å bruke $\Sigma$ for å regne $s_n$

\begin{align*}
    s_n &= \sum_{i=1}^{n} 4(i-1)+5 \\
    &= \sum_{i=1}^{n} 4i + 1 \\
    &= 4 \sum_{i=1}^{n} i + \sum_{i=1}^{n} 1 \\
    &= 4 \frac{n(n+1)}{2} + n \\
    &= 4 \frac{n^2+n}{2} + n \\
    &= 2n^2 + 3n
\end{align*}

Derreter regnes $n_{12}$ eksplisitt

\begin{align*}
    s_{12} &= 2 \cdot 12^2 + 3 \cdot 12 \\
    &= 2 \cdot 144 + 36 \\
    &= 324
\end{align*}

\subsubsection{Hva er summen av de n første leddene i denne rekka}

Summen for de n første leddene ble definert i oppgaven over. Men den er: $2n^2 + 3n$

\subsection{Oppgave 2: Betrakt rekka $\frac{15}{16} + \frac{3}{4} + \frac{3}{5} + \frac{12}{25}$}

\subsubsection{Begrunn hvorfor rekka er geometrisk og hvorfor den konvergerer}

Rekka er geometrisk fordi den kan defineres med $a_{n+1}=a_n k + a_1$

variabelen $k$ avgjør om rekka konvergerer og i denne rekka er $k=0.8$ som betyr at den konvergerer.

En rekke konvergerer når $k \in [0, 1]$

\subsubsection{Funn summen av rekka}

Her må jeg bruke summen av rekka og sette n mot uendelig

\subsection{Oppgave 3: Bruk fortegnsskjema til å finne konvergeringsområdet til rekka}


\subsection{Oppgave 4: Regn ut integralene}

\subsubsection{$\int_{0}^{1} (e^{4x}+2x) dx$}

\begin{align*}
    & \int_{0}^{1} (e^{4x}+2x) dx \\
    & = [\frac{1}{4} e^{4x} + x^2]^1_0 \\
    &= \frac{1}{4} e^4 + 1^2 - \frac{1}{4}1 \\
    &= \frac{1}{4} e^4 + \frac{3}{4}
\end{align*}

\subsubsection{$\int_{0}^{3} (x^2-2x-1)^3(2x-2) dx$}

Denne løsningen bruker rå makt

\begin{align*}
    & \int_{0}^{3} (x^2-2x-1)^3(2x-2) dx \\
    &= \int_{0}^{3} (x^2-2x-1)^2(x^2-2x-1)(2x-2) dx \\
    &= \int_{0}^{3} (x^4-4x^2+2x^2+4x+1)(x^2-2x-1)(2x-2) dx \\
    &= \int_{0}^{3} (x^6-6x^5+9x^4+4x^3-7x^2-8x-1)(2x-2) dx \\
    &= \int_{0}^{3} 2x^7-12x^6+18x^5+8x^4-14x^3-16x^2-2x dx \\
    &= [\frac{2}{8}x^8-\frac{14}{7}x^7+\frac{30}{6}x^6-\frac{10}{5}x^5-\frac{22}{4}x^4-\frac{2}{3}x^2+7x^2+2x]^3_0 \\
    &= [\frac{1}{4}x^8-2x^7+5x^6-2x^5-\frac{11}{2}x^4-\frac{2}{3}x^3+7x^2+2x]^3_0 \\
    &= (\frac{1}{4}*3^8-2*3^7+5*3^6-2*3^5-\frac{11}{2}*3^4-\frac{2}{3}*3^3+7*3^2+2*3) \\
    &- (\frac{1}{4}*1^8-2*1^7+5*1^6-2*1^5-\frac{11}{2}*1^4-\frac{2}{3}*1^3+7*1^2+2*1) \\
    &= (\frac{1}{4}*3^8-2*3^7+5*3^6-2*3^5-\frac{11}{2}*3^4-\frac{2}{3}*3^3+7*3^2+2*3) - \frac{49}{12} \\
    &= (\frac{1}{4}*6561- 2*2187+5*729-2*243-\frac{11}{2}*81-\frac{2}{3}*27+7*9+6) - \frac{49}{12} \\
    &= (\frac{6561}{4}- 2*2187+5*729-2*243-\frac{891}{2}-\frac{54}{3}+63+6) - \frac{49}{12} \\
    &= (\frac{6561}{4}- 4374+3645-486-\frac{891}{2}-\frac{54}{3}+63+6) - \frac{49}{12} \\
    &= \frac{80}{3}
\end{align*}

\subsubsection{$\int e^{2x}(x^2+3x+1) dx$}

Denne løsningen bruker delvis integrasjon

\begin{align*}
    & \int e^{2x}(x^2+3x+1) dx \\ 
    &= \frac{1}{2} e^{2x} (x^2+3x+1) - \int \frac{1}{2} e^{2x} (2x+3) \\
    &= \frac{1}{2} e^{2x} (x^2+3x+1) - \frac{1}{2} \int e^{2x} (2x+3) \\
    &= \frac{1}{2} e^{2x} (x^2+3x+1) - \frac{1}{2} ( \frac{1}{2} e^{2x} (2x+3) - \int \frac{1}{2} e^{2x}2) \\
    &= \frac{1}{2} e^{2x} (x^2+3x+1) - \frac{1}{2} ( \frac{1}{2} e^{2x} (2x+3) - \frac{1}{2} e^{2x}) \\
    &= \frac{1}{2} e^{2x} (x^2+3x+1) - \frac{1}{2} * \frac{1}{2} e^{2x} ((2x+3) - 1) \\
    &= \frac{1}{2} e^{2x} ((x^2+3x+1) - \frac{1}{2} ((2x+3) - 1)) \\
    &= \frac{1}{2} e^{2x} ((x^2+3x+1) - \frac{2x+2}{2}) \\
    &= \frac{1}{2} e^{2x} ((x^2+3x+1) - (x-1)) \\
    &= \frac{1}{2} e^{2x} ((x^2+2x+2)) \\ 
    &= e^{2x} (\frac{1}{2} x^2 + x + 1) + C
\end{align*}

\subsubsection{$\int \frac{3x-37}{x^2-3x-4}$}

her brukes brøkssplitting

\begin{align*}
    & \frac{3x-37}{x^2-3x-4} \\
    &= \frac{A}{x+1} + \frac{B}{x-4} \\
    &= \frac{A(x-4)}{(x+1)(x-4)} + \frac{B(x+1)}{(x+1)(x-4)} \\
    &= A(x-4) + B(x+1) = 3x - 37 \\
    & x=4 \rightarrow B(5) = 15 - 37 \\
    & B = \frac{-22}{5} \\
    & x = -1 \rightarrow -5A = -15 -37 \\
    &= A = \frac{52}{5}
\end{align*}

Nå som $A$ og $B$ er definert kan vi regne ut integralene

\begin{align*}
    & \int \frac{\frac{52}{5}}{\frac{(x-1)}{1}} dx + \int \frac{\frac{-22}{5}}{\frac{(x-4)}{1}} \\
    &= \int \frac{52}{5} \div \frac{(x+1)}{1} + \int \frac{-22}{5} \div \frac{x-4}{1} \\
    &= \int \frac{52}{5(x+1)} + \int \frac{-22}{5(x-4)} \\
    &= \frac{52}{5} \int \frac{1}{(x+1)} + \frac{-22}{5} \int \frac{1}{(x-4)} \\
    &= \frac{52}{5} (ln|x| + x) - \frac{22}{5} (ln|x| - \frac{1}{4}{x}) \\
    &= \frac{52(ln|x| + x)}{5}  - \frac{22(ln|x| - \frac{1}{4}{x})}{5} \\
    &= \frac{52(ln|x| + x) - 22(ln|x| - \frac{1}{4}{x})}{5} \\
    &= \frac{52ln|x| + 52x - 22ln|x| - \frac{-22}{4}{x}}{5} \\
    &= \frac{30ln|x| + \frac{230}{4}x}{5} \\
    &= \frac{30ln|x| + \frac{115}{2}x}{5} \\
    &= 6ln|x| + \frac{575}{2}x+C
\end{align*}

\subsubsection{$\int 8xe^{2x^2}dx$}

\begin{align*}
    & \int 8xe^{2x^2}dx \\
    &= \int 8xe^{u} \frac{du}{4x} \\
    &= 2 \int e^{u} \\ 
    &= 2e^{2x^2} + C
\end{align*}

\subsection{Oppgave 5: La $f(x)=5x^2+3, D_f=[-1, 2]$ Hva er volumet av omdreiningslegemet som framkommer ved å dreie grafen til $f 360^\circ$ om x-aksen?}

For å finne omdreiningslegemet må integralet $\int_{-1}^{2} 5x^2+3$ løses

\begin{align*}
    V &= \int_{-1}^{2} 5x^2+3 \\
    &= [\frac{5}{3}x^3+3x]^{2}_{-1} \\
    &= \frac{5}{3}2^3+6-(\frac{5}{3}*(-1)^3-3) \\
    &= \frac{5}{3}8 + 6 + \frac{5}{3} + 3 \\
    &= 26
\end{align*}

\subsection{Oppgave 6: Regn ut itegral}

funksjonen er 3(x+1)(x-2) fordi den krysser -1 og 2 i x aksen.

For å finne arealet av det avgrenset området regnes det i tre separate integraler:

\begin{align*}
    &= \int_{-3}^{-1}3(x+1)(x-2)dx  + \int_{-1}^{2}3(x+1)(x-2)dx + \int_{2}^{3}3(x+1)(x-2)dx \\
    &= 3 (\int_{-3}^{-1}(x+1)(x-2)dx  + \int_{-1}^{2}(x+1)(x-2)dx + \int_{2}^{3}(x+1)(x-2)dx) \\
    &= 3 (\int_{-3}^{-1}(x^2-x-2) dx  + \int_{-1}^{2}(x^2-x-2) dx + \int_{2}^{3}(x^2-x-2) dx) \\
    &= 3 (\int_{-3}^{-1}(x^2-x-2) dx  + \int_{-1}^{2}(x^2-x-2) dx + \int_{2}^{3}(x^2-x-2) dx) \\
    &= 3 ([\frac{1}{3}x^3 - \frac{1}{2} x^2 - 2x]^{-1}_{-3} + [\frac{1}{3}x^3 - \frac{1}{2} x^2 - 2x]^2_{-1} + [\frac{1}{3}x^3 - \frac{1}{2} x^2 - 2x]^3_2) \\
    &= 3 (((\frac{1}{3}(-1)^3 - \frac{1}{2} (-1)^2 - 2(-1)) - (\frac{1}{3}(-3)^3 - \frac{1}{2} (-3)^2 - 2(-3))) \\
    &+ ((\frac{1}{3}(2)^3 - \frac{1}{2} (2)^2 - 2(2)) - (\frac{1}{3}(-1)^3 - \frac{1}{2} (-1)^2 - 2(-1))) \\
    &+ ((\frac{1}{3}(3)^3 - \frac{1}{2} (3)^2 - 2(3)) - (\frac{1}{3}(2)^3) - \frac{1}{2} (2)^2 - 2(2))) \\
    &= 3 (((\frac{-1}{3}-\frac{1}{2}+2) - (\frac{-27}{3}-\frac{9}{2}+6)) \\
    &+ ((\frac{8}{3}-\frac{4}{2}-4) - (\frac{-1}{3})-\frac{1}{2}+2) \\
    &+ ((\frac{27}{3} - \frac{9}{2} - 6) - (\frac{8}{3}-\frac{4}{2}-4))) \\
\end{align*}


\subsection{Direkte}

\subsubsection{Bevis at $n^3+n, n \in \mathbb{N}$ er et partall direkte}


vi vil vise at $n^3+n$ er et partall, det vil si at det fins $m \in \mathbb{N}$ slik at $n^3+n=2m$
Jeg skal bevise det for når $n$ er et partall og oddetall

Først antar vi at $n$ er et partall

da fins $k \in \mathbb{N}$ slik at $n=2k$. Vi får da at $n^3+n=(2k)^3+2k=2^3 \cdot k^3+2k=8k^3+2k=2(4k^3+k)$.

Siden $k \in \mathbb{N}$ er $2(4k^3+k) \in \mathbb{N}$

Dermed fins $m \in \mathbb{N}$ nemelig $m=4k^3+k$ slik at $n^3+n=2m$

Så antar vi at $n$ er et oddetall

da fins $k \in \mathbb{N}$ slik at $n=2k+1$. Vi får da at $n^3+n=(2k+1)^3+2k+1=(4k^2+4k+1)(2k+1)+2k+1=8k^3+8k^2+2k+4k^2+4k+1+2k+1=8k^3+12k^2+8k+2=2(4k^3+6k^2+4k+1)$

samme som i sta finnes $m \in \mathbb{N}$ er også, nemelig $m=4k^3+6k^2+4k+2$ slik at $n^3+n=2m$ også når $n$ er et oddetall

Så da har vi besvist at $n^3+n$ er et partall for alle heltall i n.

\paragraph{riktig men}

Til neste bevis med flere steg bruk paragraph for hver seksjon, også ta utregning i \ [ ] \

\subsubsection{Bevis at $a,b,c \in \mathbb{Z}$ hvis a går opp i b, og b går opp i c, så vil a gå opp i c }

vi antar at $\frac{a}{b} \in \mathbb{Z}$ og $\frac{b}{c} \in \mathbb{Z}$

derfor fins $k, u \in \mathbb{Z}$ slik at $\frac{a}{b}=k$ og $\frac{b}{c}=u$

vi får da at:

\begin{align*}
    \frac{a}{b} &= k \rightarrow a = kb \\
    \frac{b}{c} &= u \rightarrow b = uc \\
    \frac{a}{c} &= \frac{kb}{c} = \frac{kuc}{c}=ku \\
    \frac{a}{c}=ku \in \mathbb{Z}
\end{align*}

siden $k,u \in \mathbb{Z}$ og $\frac{a}{c}=ku$ har vi bevist at $\frac{a}{c} \in \mathbb{Z}$

\paragraph{Nesten}

dette fungerer men er litt feil. a går opp i b betyr egentlig $b \mid a$
og jeg skulle ha brukt $b \mid a$ i stedet for $\frac{b}{a}$
legg også til \textbf{Q.E.D} på slutten for at beviset skal bli mer proft

\subsubsection{Bevis at $n \in \mathbb{Z}, n \geq 5$, så kan ikke alle tallene $n, n+2, n+4$ være primtall}

Vi lar $\mathbb{P}$ betegne mengden av alle primtall

\paragraph{Tilfelle 1: $n$ er et partall}

Det finnes $k \in \mathbb{Z}$ slik at $n=2k$

Det gir oss $\frac{2k}{2} \in \mathbb{Z}$ og dermed $n \notin \mathbb{P}$

\paragraph{Tilfelle 2: $n$ er et odetall}

Det finnes $k \in \mathbb{Z}$ slik at $n=2k+1$

\subsection{Induksjon}

\subsubsection{Bevis at $2+4+6+...+2n=n(n+1)$}

$P(1): 2=1(1+1) \rightarrow 2=2$ P(1) stemmer 

Jeg antar at utsagnet er sant for alle naturlige tall $n=k \in \mathbb{N}$

$2+4+6+...+2k=k(k+1)$

Jeg vil vise at utsagnet $n = k + 1$ er sant:

\begin{align*}
    2+4+6+...+2k+2(k+1) &= (k+1)((k+1) + 1) \\
    2+4+6+...+2k+2(k+1) &= k(k+1)+2(k+1) \\
    &= k^2+3k+2 \\
    &= (k+1)(k+2)
\end{align*}

P(N) stemmer

Dermed har jeg bevist ved induksjon at utsagnet er sant for alle naturlige tall $n$

\textbf{Q.E.D}

\textbf{Riktig}
Funker kan legge til:
\[
\sum_{i=1}^{n} 2i = n(n + 1)
\]
For at det skal bli penere

\subsubsection{Bevis $a_n=\frac{3}{2} \cdot 2^n -1$ er en eksplisitt formel for $a_1=2$ og $a_{n+1}=2a_n+1$ for $n \ge 1$}

Jeg sjekker $n=1$

$a_1=\frac{3}{2} \cdot 2 - 1 = 2$

P(1) stemmer

Jeg antar at formelen gjelder for et naturlig tall $k$, det vil si  $a_k=\frac{3}{2} \cdot 2^k - 1$

Jeg skal vise at vi da har $a_{k+1} = \frac{3}{2} \cdot 2^{k+1} - 1$

\begin{align*}
    a_{k+1} &= 2a_k+1 \\
    &= 2(\frac{3}{2} \cdot 2^{k} - 1) + 1 \\
    &= 3 \cdot 2^k - 2 + 1 \\
    &= \frac{3}{2} \cdot 2^{k+1} - 1
\end{align*}

Dermed har jeg bevist at formlen gjelder for alle naturlige tall

\textbf{Q.E.D}

\textbf{Riktig}

\subsubsection{Bevis at $P(n): (x^n)'= nx^{n-1}$ for alle $n \in \mathbb{N}$}

Jeg sjekker $n=1$

$(x^1)'= 1x^{1-1} \rightarrow x'=1$

Vi vet at $x'$ har et stigningstall på $1$, det vil si $x'= 1$

$P(1)$ stemmer

Jeg antar at utsagnet gjelder for alle naturle tall $k \in \mathbb{N}$, Det vil si $(x^k)' = kx^{k-1}$

Jeg vil vise at $P(k+1)$ er sant:

$$(x^{k+1})' = (k+1)x^{(k+1)-1} = x^k(k+1)$$

Jeg bruker definisjonen til den deriverte

\begin{align*}
    (x^{k+1})'=\lim_{\Delta x \rightarrow 0}\frac{(\Delta x + x)^{k+1}-x^{k+1}}{\Delta x}
\end{align*}

For å løse denne oppgaven må jeg bruke binomisk formel, den sier at:

$$(a+b)^n=\sum_{i=0}^{n}\binom{n}{i}a^{n-i}b^i$$
\begin{align*}
    (x^{k+1})' &= \lim_{\Delta x \rightarrow 0} \frac{(\Delta x + x)^{k+1} - x^{k+1}}{\Delta x} \\
    (\Delta x + x)^{k+1} &= \sum_{i=0}^{k+1}\binom{k+1}{i}x^{k+1-i}\Delta^i \\
    (\Delta x + x)^{k+1} &= \binom{k+1}{1}x^{k+1-1}\Delta x^1+\binom{k+1}{2}x^{k-1}\Delta x^2+...+\binom{k+1}{k+1}x^{(k+1)-(k+1)} \Delta x ^{k+1} \\
    (\Delta x + x)^{k+1} &= x^{k+1}+\binom{k+1}{1}x^k\Delta x + \binom{k+1}{2}x^{k-1}{\Delta x} ^2+...+{\Delta x}^{k+1} \\
    (x^{k+1})' &= \lim_{\Delta x \rightarrow 0} \frac{[x^{k+1}+\binom{k+1}{1}x^k\Delta x + \binom{k+1}{2}x^{k-1}{\Delta x} ^2+...+{\Delta x}^{k+1}] - x^{k+1}}{\Delta x} \\
    &= \lim_{\Delta x \rightarrow 0} \frac{\binom{k+1}{1}x^k\Delta x + \binom{k+1}{2}x^{k-1}{\Delta x} ^2+...+{\Delta x}^{k+1}}{\Delta x} \\
    &= \lim_{\Delta x \rightarrow 0} \binom{k+1}{1}x^k + \binom{k+1}{2} x^{k-1} \Delta x + ... + {\Delta x}^k \\
    &= \binom{k+1}{1}x^k+\binom{k+1}{2}x^{k-1}0+...+0^k \\
    &= \binom{k+1}{1}x^k \\
    &= (k+1)x^k \\
\end{align*}

Dermed har jeg bevist at $(x^{k+1})=(k+1)x^k$ som var det jeg vile bevise. Dette viser at utsagnet gjelder for alle natulrige tall $n \in \mathbb{N}$

\textbf{Q.E.D}

\textbf{Riktig}

Brukte chat gpt da, men bare til binomisk formel. Kunne også ha blitt løst med potensregelen.

\newpage

\subsubsection{Bevis at $n^5-n$ er delelig med $5$ for $n \in \mathbb{N}$}

$P(1): 1^5-1=0$

$P(1)$ stemmer fordi $\frac{0}{5}\in \mathbb{Z}$

Jeg antar at utsagnet er sant for $k \in \mathbb{N}$ og er delig med 5. Det vil si at det fines $s \in \mathbb{Z}$ slik at $k^5-k=5s$

Jeg vil vise at $P(n+1)$ er sant, det vil si at det finnes tall $r \in \mathbb{Z}$ slik at $(k+1)^5-(k+1)=5r$

\begin{align*}
    (k+1)^5-(k+1) &= 1k^5+5k^4+10k^3+10k^2+5k^1+1-k-1 \\
    &= \textcolor{red}{k^5-k}+5k^4+10k^3+10k^2+5k \\
    &= \textcolor{red}{5r}+5k^4+10k^3+10k^2+5k \qquad \textbf{Her bruker vi antagelsen}\\
    &= 5(s+k^4+2k^3+2k^2+k) \\
    &= 5(r)
\end{align*}

I den siste likheten har jeg kalt $s+k^4+2k^3+2k^2+k$ for $r$

Jeg har dermed bevist med induksjon at $\frac{n^5-n}{5} \in \mathbb{Z}$ for $n \in \mathbb{N}$

\textbf{Riktig}
Men kan kanskje bruke mer ord i sluttsetningen

\subsubsection{Lag en tabel som viser hvilke naturlige tall ser ut til at $2^n > 2n$ så bevis resultatet ved induksjon}

\begin{tabular}{|c|c|c|c|}
    \hline
    $n$ & $2^n$ & $2n$ & $2^n > 2n$ \\
    \hline
    1 & 2 & 2 & usant \\
    2 & 4 & 4 & usant \\
    3 & 8 & 6 & sant \\
    4 & 16 & 8 & sant \\
    5 & 32 & 10 & sant \\
    6 & 64 & 12 & sant \\
    \hline
\end{tabular}

Jeg ønske å bevise at $2^n > 2n$ for $n \ge 3$

Utsagnet er sant for $n=3$. Jeg antar at utsagnet er sant for et naturlig tall $k \ge 3$, altså at $\textcolor{red}{2^k > 2k}$

Jeg må vise at da er også $2^{k+1} > (2(k+1))$

\begin{align*}
2^{k+1} &= 2 \cdot \textcolor{red}{2^k} \\
&> 2 \cdot \textcolor{red}{2k} \qquad \textbf{Her bruker vi antagelsen} \\
&= 2(k+1)
\end{align*}

Jeg har dermed bevist ved induksjon at $2^n > 2n$ er sant for alle naturlige tall $n \ge 3$

\textbf{Riktig}

\bibliographystyle{alpha}
\bibliography{sample}

\end{document}
