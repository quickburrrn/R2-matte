\documentclass{article}

% Language setting
% Replace `english' with e.g. `spanish' to change the document language
\usepackage[english]{babel}

% Set page size and margins
% Replace `letterpaper' with `a4paper' for UK/EU standard size
\usepackage[letterpaper,top=2cm,bottom=2cm,left=3cm,right=3cm,marginparwidth=1.75cm]{geometry}

% Useful packages
\usepackage{amsmath}
\usepackage{amssymb}
\usepackage{graphicx}
\usepackage{inconsolata}
\usepackage{minted}
\usepackage[colorlinks=true, allcolors=blue]{hyperref}

\title{R2 Eksamen}
\author{Skrevet av André Hansen}

\begin{document}
\maketitle

\begin{abstract}
Dette er en template til selve eksamen
\end{abstract}

\section{Del 1: Uten hjelpemidler}

\subsection{Oppgave 1: Regn ut integralene}

\subsubsection{a) $\int_{0}^{1}(4x^2+3)dx$}

$$\int_{0}^{1}(4x^2+3)dx=[\frac{4}{3}x^3+3x]^1_0=\frac{4}{3}+3=\frac{13}{3}$$

\subsubsection{b) $4x\sqrt{x^2+2}dx$}

\text{Ikke nokk integrajonskunskaper enda :(}

\subsection{Oppgave 2}

\subsubsection{a)}

vi har : $\vec{n}=[-1,1,1]$

kryssprodukutet av $\vec{u}$ og $\vec{v}$ vil gi oss en normalvektor til $\alpha$ 

\begin{align*}
    & \quad \vec{u} \times \vec{v} \\
    &= [1, 2, -1] \times [-1, 1, -2] \\
    &= [2 \cdot -2 - (-1) \cdot 1, -1 \cdot 1 - 1 \cdot -2, 1 \cdot 1 - 2 \cdot -1] \\
    &= [-3, 3, 3] \parallel \vec{n} \\
\end{align*}

\subsubsection{b)}

Vi vet et punkt $P$ og normalvektor $\vec{n}$

Dermed kan vi finne $d$ komponenten av $\alpha$

\begin{align*}
    \alpha &= ax+by+cy+d=0 \\
    &= -1 \cdot 2 + 1 \cdot 0 + 1 \cdot 1 + d = 0 \\
    &= d=1 \\ 
    \alpha &= -x+y+z+1=0
\end{align*}

\subsection{Oppgave 3}

Ligningen kan løses med identiteten: $sin^2x+cos^2x=1$

\begin{align*}
    \cos^2x-3\sin^2x &= -2 , \quad x \in [0, 2\pi) \\
    \cos^2x-3(1-\cos^2x) &= -2 \\
    4\cos^2x-3 &= -2 \\
    4\cos^2x &= 1 \\
    \sqrt{4\cos^2x} &= \sqrt{1} \\
    2\cos x &= 1 \\
    \cos x &= \frac{1}{2} \\
    L_{cos} &= \{\frac{\pi}{3}, \frac{5\pi}{3}\} \\
    \text{Så for } \sin \\
    \cos^2x-3\sin^2x &= -2 , \quad x \in [0, 2\pi) \\
    1 - \sin^2x - 3 \sin^2x &= -2 \\
    -4\sin^2x &= -3 \\
    \sin^2x &= \frac{3}{4} \\
    \sin x &= \frac{\sqrt{3}}{2} \\
    L_{sin} &= \{\frac{\pi}{3}, \frac{4\pi}{3}\} \\
    L = L_{cos} &\cup L_{sin} = \{\frac{\pi}{3}, \frac{4\pi}{3}, \frac{5\pi}{3}\}
\end{align*}

\subsection{Oppgave 4}

vi har : $S_n=3n^2+4n$

ved å regne $s_4-s_3$ får vi $a_4$

\begin{align*}
    &s_4 - s_3 \\
    &= 3 \cdot 4^2 + 4 \cdot 4 - 3 \cdot 3^2 - 4 \cdot 3 \\
    &= 25 \\ 
    &= a_4 
\end{align*}

\subsection{Oppgave 5}

Etter mye prøving og feiling kom jeg fram til at den eksplisitte formelen til $b_n$ er:

$$b_n=2\cdot \frac{2}{3}^{n-1}$$

Dermed kan vi finne $s_4$ ved å regne $s_3+b_4$

\begin{align*}
    & s_3 + b_4 \\
    &= \frac{38}{9} + 2 \cdot \frac{2}{3}^{4-1} \\
    &= \frac{38}{9} + \frac{16}{27} \\
    &= \frac{130}{27}
\end{align*}

\subsection{Oppgave 6}

\subsubsection{a)}

Eleven ønsker å regne integralet $\int_{0}^{\pi} 2\sin (x+\frac{\pi}{6})$

\subsubsection{b)}

\begin{align*}
    & \int_{0}^{\pi} 2\sin (x+\frac{\pi}{6}) \\
    &= 2 [\cos (x+\frac{\pi}{6})]^\pi_0 \\
    &= 2 \cos \frac{7\pi}{6} \\
    &= -\sqrt{3}
\end{align*}

\subsection{Oppgave 7}

\subsubsection{a)}

\begin{align*}
    & [\frac{1}{1-x}+C_1]' \\
    &= \frac{1 \cdot (1-x) - x \cdot -1}{(1-x)^2} \\
    &= \frac{1 - x + x}{(1-x)^2} \\
    &= \frac{1}{(1-x)^2}
\end{align*}

\subsubsection{b)}

\begin{align*}
    & \int \frac{1}{(1-x)^2}dx \\
    &= \frac{u}{(u)^2} , \quad u = 1-x \\
    &= \frac{1}{1-x}+C_2
\end{align*}

\section{Del 2: Med hjelpemidler}

\subsection{Oppgave 1}

\subsubsection{a)}

\begin{figure}[H]
    \centering
    \includegraphics[width=0.7\textwidth]{bilder/oppgave1a1.png}
    \caption{her har jeg utført en regresjonsanalyse i geogebra}
    \includegraphics[width=0.7\textwidth]{bilder/oppgave1a2.png}
    \caption{Den periodiske formelen er : $0,475 + 0,485 \sin (0,212 x - 2,92)$}
\end{figure}

\subsubsection{b)}

Hvis vi har den periodiske formelen på formen : $b+A\sin(c-\phi)$

Vet vi at formelen for perioden er : $\frac{2 \pi}{c}$

Dermed får vi :

$$p=\frac{2 \pi}{0,212} \approx 29,6$$

Det er samme tid mellom et punkt fra en periode til neste periodes dermed kan vi at det tar ca $29,6$ døgn mellom hver fullmåne

\begin{figure}[H]
    \centering
    \includegraphics[width=0.7\textwidth]{bilder/oppgave1b1.png}
    \caption{utregning i CAS}
\end{figure}

\subsubsection{c)}

Her antar jeg at observasjonene begynte på starten av året.

Ved å løse $g(x)=0,5$ finner jeg ut hva hvilke dager halvmånen begynner å slutter

\begin{figure}[H]
    \centering
    \includegraphics[width=0.7\textwidth]{bilder/oppgave1c1.png}
    \caption{Utregning i cas}
\end{figure}

Fra døgn 14 til døgn 28 var minst halve månen synlig

\subsection{Oppgave 2}

\subsubsection{a)}

For å finne området på intervalet hvor $S(x)$ konvergerer finner vi alle punkter hvor $S(x)\ne \infty$

Først skriver jeg om $S(x)$ ved formelen for uendelige geometriske rekker: $S(x)=\frac{-1}{a-1}$

\begin{align*}
    S(x) &= \cos x + 2 \sin x \cos x + 4 \sin^2 x \cos x + ... \\
    &= \lim_{n \rightarrow \infty } \cos x \sum_{i=0}^{n} 2^i \sin^i x \\
    &= \lim_{n \rightarrow \infty } \cos x \sum_{i=0}^{n} (2\sin)^i x \\
    &= \frac{-\cos x}{2 \sin x - 1} \\
    & \text{Finner nullpunktene til nevner for å finne udefinerte punkter til $S(x)$} \\
    &= 2 \sin x - 1 = 0 \\
    &= \sin x = \frac{1}{2} \\
    &= x = \frac{\pi}{6} +k2\pi \vee x = \frac{\pi5}{6} + k2\pi \\
    & L = \{ - \frac{11 \pi}{6} , - \frac{7 \pi}{6} , \frac{\pi}{6}, \frac{5\pi}{6}\}
\end{align*}

I området $[-2\pi, 2\pi]$ konvergerer rekken på $x \in \{ - \frac{11 \pi}{6} , - \frac{7 \pi}{6} , \frac{\pi}{6}, \frac{5\pi}{6}\}$ 

\subsubsection{b)}

Siden $\lim_{x \rightarrow \frac{\pi}{6}^+} S(x) = \infty$ og $\lim_{x \rightarrow \frac{\pi}{6}^-} S(x) = -\infty$

Betyr dette at $V_S = [-\infty, \infty]$ eller $V_S=\mathbb{R}$ på intervallet $[-\pi, \pi]$

Med det kan vi si for $b=\mathbb{R}$ har $S(x)=b$ en eller flere løsninger

\subsection{Oppgave 3}

\begin{verbatim}
# definerer radius
r = 3

# Det er bare definert verdier på heltallige punkter i aksene, så lager en liste med alle punkter det er løsninger på
da = []
for n in range(-r, r+1):
    da.append(n)

# definerer en liste som skal samle alle løsninger
valid_losning = []

# kjører en nøsted løkke for å gå gjennomm alle punkter for å se om det er en riktig løsning
for x in da:
    for y in da:
        for z in da:
            # dersom det stemmer settes løsningen i listne med løsninger
            if x**2 + y**2 + z**2 == 9:
                valid_losning.append([x, y, z])

# printer alle løsninger
for n in valid_losning:
    print(f"({n[0]}, {n[1]}, {n[2]})")

\end{verbatim}

Resultat av å kjøre koden:

\begin{verbatim}
(-3, 0, 0)
(-2, -2, -1)
(-2, -2, 1)
(-2, -1, -2)
(-2, -1, 2)
(-2, 1, -2)
(-2, 1, 2)
(-2, 2, -1)
(-2, 2, 1)
(-1, -2, -2)
(-1, -2, 2)
(-1, 2, -2)
(-1, 2, 2)
(0, -3, 0)
(0, 0, -3)
(0, 0, 3)
(0, 3, 0)
(1, -2, -2)
(1, -2, 2)
(1, 2, -2)
(1, 2, 2)
(2, -2, -1)
(2, -2, 1)
(2, -1, -2)
(2, -1, 2)
(2, 1, -2)
(2, 1, 2)
(2, 2, -1)
(2, 2, 1)
(3, 0, 0)
\end{verbatim}

\begin{figure}[H]
    \centering
    \includegraphics[width=0.7\textwidth]{bilder/oppgave3.png}
    \caption{Alle heltallige løsninger i Geogebra}
\end{figure}

\subsection{Oppgave 4}

\subsubsection{a)}

Jeg løste oppgaven ved å definere en kurve i geogebra derivere den og regne avstanden til velositetsvektoren

\begin{figure}[H]
    \centering
    \includegraphics[width=0.7\textwidth]{bilder/oppgave4.png}
    \caption{Løsninge i CAS}
\end{figure}

\subsubsection{b)}

Grunnen til at $\vec{r}'(t) \perp [3, 1, 5]$ er fordi at $[3, 1, 5]$ er normalvektoren til planet

For at et punkt skal tangere må $\vec{r}'(t)$ være parallel med planet dermed $\vec{r}'(t) \cdot n = 0$

\subsubsection{c)}

Jeg fant kordinatene ved å løse ligningen $\vec{r}'(t) \cdot [3, 1, 5] = 0 $ og sette resultate i $\vec{r}(t)$

\begin{figure}[H]
    \centering
    \includegraphics[width=0.7\textwidth]{bilder/oppgave41.png}
    \caption{Kordinatene til punktene}
\end{figure}

\begin{figure}[H]
    \centering
    \includegraphics[width=1.1\textwidth]{bilder/oppgave42.png}
    \caption{Utregning i Geogebra}
\end{figure}

\subsection{Oppgave 5}

\subsubsection{a)}

Det bil ta syv steg for å flyttet diskene dermed : $F(3)=7$

\subsection{b)}

Den rekursive sammenhengen mellom $F(n)$ og $F(n-1)$ er:

\begin{align*}
    F(n) &= 2 \cdot F(n-1) + 1, \quad F(1) = 1 \\
    &\text{eller} \\
    F(n) &= F(n-1) + 2^n, \quad F(1)=1
\end{align*}

\begin{align*}
    F(1) = 1 \\
    F(2) = 3 \\
    F(3) = 7 \\
    F(4) = 15 \\
    F(5) = 31 \\
    F(6) = 63 \\
    F(7) = 127 \\
    F(8) = 255 \\
    F(9) = 511 \\ 
    F(10) = 1023 \\
\end{align*}

\subsubsection{c)}

Den rekursive formelen er gitt ved $a_n+2^n, \quad a_1 = 1$

Den kan skrives om på sum form : $\sum_{i=0}^{n} 2^i$

Derreter kan vi bruke identiteten: $\sum_{i=1}^{n} a^i = \frac{a^{n+1} - 1}{a - 1}$, for å skrive det på samlet form :

$F(n)=2^n - 1$

\subsubsection{d)}

$P(1) : 2^0 = 2^1-1 \rightarrow 1 = 1$

$P(1)$ stemmer

Jeg antar at utsaget er sant for $n = k \in \mathbb{N}$

$2^0+2^1+2^2+...+2^{k-1} = 2^k-1$

For å bevise dette må utsaget $n=k+1$ også være sant:

\begin{align*}
    2^0+2^1+2^2+...+2^{k-1}+2^{k} &= 2^{k+1}-1 \\
    &= 2^k-1 + 2^k \quad \text{bruker antagelse} \\
    &= 2 \cdot 2^k - 1 \\
    &= 2^{k+1} - 1 \\
\end{align*}

$P(n)$ stemmer

\textbf{Q.E.D}

\subsection{Oppgave 6}

\subsubsection{a)}

For å finne  ut hvor mye vann som renner over finner vi volumet til delen av kulen i glasset

For å finne volumet til en del av sirkelen regner vi det som omdreiningslegemet:

$n$ er hvor langt kulen er i glasset: $\pi \int_{0}^{n} \sqrt{r^2 - (x - r)^2}$

For å finne hvor langt kule er i glasset finner vi først hvor langt det er fra bunnen av glasset til senter av kulen:

$\sqrt{r^2 + (r \cdot \tan(55^{\circ}))^2}$

$55^{\circ}$ fordi $180^{\circ} - 90^{\circ} - 35^{\circ} = 55^{\circ}$

Så kan vi finne lengden av sirkelen i glasset ved

$r - (\sqrt(r^2 + (r \tan(55^{\circ}))^2) - 10)$

Dette subtraherer radius med forskjellen fra bunnen av glasset til sentrum av kulen og bunnen av glasset til toppen av glasset

\begin{figure}[H]
    \centering
    \includegraphics[width=0.7\textwidth]{bilder/oppgave6.png}
    \caption{Utregning i Geogebra}
\end{figure}

Det vil renne over ca $390 cm^3$ vann over dersom $R=7cm$

\subsubsection{b)}

For å finne $R$ slik at det er mest mulig vann som regner ut av glasset, hvor kulen tar mest plass i glasset.

Jeg lager en funksjon for kulesegmentet i glasset.

For å finne arealet til et kulesegment av kulen bruker jeg funksjonen:

$f(x) = \pi \int_{0}^{mengde} \sqrt{R^2 - (x-R)^2}$

Enten er hele kulen i glasset eller så er en liten del av kulen i glasset.

Hvis slutten av kulen er over glasset må vi regne volumet av kulesegmentet som er inni. Hvis ikke regner vi volumet til hele kulen.

Vi kan finne slutten av kulen ved å sumere avstanden fra bunnen av glasset med radius:

$Slutt(r) = r + \sqrt{r^2 + (r \cdot \tan(55^{\circ}))^2}$

Vi kan regne avstanden fra toppen av glasset og til bunnen av kulen:

$Segment(r) = r - (\sqrt{x^2 + x \cdot \tan(55^{\circ})} - 10) = r + 10 - \sqrt{x^2 + x \tan(55^{\circ})}$

Dermed kan vi lage en funksjon for mengden av kulen som skal integreres:

\[
Menge(r)=
\begin{cases}
    Segment(r), & \text{hvis } Slutt(x) > 10, \\
    2r, & \text{ellers}.
\end{cases} \\
\]

eller

\[
Menge(r) = 
\begin{cases}
    r + 10 - \sqrt{r^2 + (r \tan(55^{\circ}))^2}, & \text{hvis } r + \sqrt{r^2 + (r \tan (55^{\circ}) )} > 10, \\
    2r, & \text{ellers}.
\end{cases} \\
\]

Dermed kan vi finne volumet kulen tar i glassed ved:

$V(R, y) = \pi \int_{0}^{Menge(R)} (\sqrt{R^2 - (x - R)^2})^2$

Dersom vi ser på grafen $V$ ser vi at den har topunkt når den bytter funksjon

Dermed kan vi finne $R$ ved:

\begin{align*}
    x+\sqrt{x^2 + (x \cdot \tan(55^{\circ}))^2} &= 10 \\
    x &\approx 3,65
\end{align*}

$R$ må være ca $3,65$ for at mest mulig vann renner ut av glasset.

\begin{figure}[H]
    \centering
    \includegraphics[width=1.1\textwidth]{bilder/oppgave62.png}
    \caption{Utregning i Geogebra}
\end{figure}

\bibliographystyle{alpha}
\bibliography{sample}

\end{document}
