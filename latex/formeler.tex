\documentclass{article}

% Language setting
% Replace `english' with e.g. `spanish' to change the document language
\usepackage[english]{babel}

% Set page size and margins
% Replace `letterpaper' with `a4paper' for UK/EU standard size
\usepackage[letterpaper,top=2cm,bottom=2cm,left=3cm,right=3cm,marginparwidth=1.75cm]{geometry}

% Useful packages
\usepackage{amsmath}
\usepackage{amssymb}
\usepackage{graphicx}
\usepackage{inconsolata}
\usepackage{minted}
\usepackage[colorlinks=true, allcolors=blue]{hyperref}

\title{Syke formeler}
\author{Skrevet av André Hansen}

\begin{document}
\maketitle

\begin{abstract}
Dette er et dokument med syke formeler og bevis jeg finner underveis
\end{abstract}

\section{Avstander}

\subsection{Avstand mellom punkt og plan}

$\alpha$ er planet gitt ved $ax+by+cz+d+0$

$\vec{n}$ er normalvektor til planet $\alpha$

$P(x_1,y_1,z_1)$ er punktet vi skal finne avstand til

$Q(x_0, y_0, z_0)$ er punkt på $\alpha$ nærmest $P$

\begin{align*}
    |\vec{PQ}|=\frac{|ax_1+by_1+cz_1+d|}{\sqrt{a^2+b^2+c^2}} \\
    \text{eller vektorbasert} \\
    |\vec{QP}|=\pm=\frac{\vec{n}\cdot\vec{QP}}{|\vec{n}|}=\frac{|\vec{n}\cdot\vec{QP}|}{|\vec{n}|} \\
\end{align*}

\text{Formelen baserer seg på at vinkelen mellom de er enten 0 eller 180}

\text{Dermed er altid cos -1 eller 1 når man regner prikkproduktet}

\text{siden $Q$ er på planet $\alpha$ kan det skrives som:}

$ax_0+by_0+cz_0+d=0 \Leftrightarrow d=-ax_0-by_0-cz_0$

\bibliographystyle{alpha}

\end{document}
