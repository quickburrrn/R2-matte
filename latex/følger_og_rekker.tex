\documentclass[12pt,a4paper]{article}

% --- Pakker ---
\usepackage[utf8]{inputenc}     % For å støtte ÆØÅ og norsk tastatur
\usepackage[T1]{fontenc}        % Riktig fontkoding
\usepackage[norsk]{babel}       % Norsk språk
\usepackage{amsmath, amssymb}   % For matematiske symboler og ligninger
\usepackage{lmodern}            % Penere font
\usepackage{geometry}           % Justering av marger
\geometry{margin=2.5cm}
\usepackage{fancyhdr}           % Penere topp- og bunntekst
\usepackage{setspace}           % Justere linjeavstand
\usepackage{graphicx}           % Bilder
\usepackage{hyperref}           % Klikkbare lenker

% --- Header/Footer ---
\pagestyle{fancy}
\fancyhf{}
\fancyhead[L]{Mitt Dokument}
\fancyhead[R]{\today}
\fancyfoot[C]{\thepage}

% --- Dokumentstart ---
\title{Følger og rekker \LaTeX}
\author{André Hansen}
\date{\today}

\begin{document}

\maketitle
\onehalfspacing  % 1.5 linjeavstand

\section*{Innledning}
Dette er dokument for å regne ut enkle ting i første kapitell

\section*{Finne eksplsitt formel til tallfølge }

Først

\begin{align}
\{a_i\}^n_{i=1} = [a_1,a_2,a_3,\ldots,a_n] \\
\ {a_i}=f(i)=eksplisitt formel
\end{align}

Så for å finne eksplsitt med linær regersjon kan man:

\begin{align}
\ {\Delta a_i=k} \\
\ \hat{y}=ax+b \\
\ \bar{x}=\frac{1}{N}\sum_{N}^{i=1} x_i \\
\ \bar{y}=\frac{1}{N}\sum_{N}^{i=1} y_i \\
\ a=\frac{\sum(x_i-\bar{x}(y_i-\bar{y}))}{\sum(x_i-\bar{x})^2} \\
\ b=\bar{y}-a\bar{x}
\end{align}

Men siden man forventer in linær funksjon er dette overkill. I stedet kan man:

\begin{align}
\ \Delta a_i=a=a_2-a_1 \\
\ a_n=\Delta a_i n + b \\
\ b=a_1-\Delta a_i \\
\ \text{full formel blir da} \\
\ a_n=(a_2-a_1)n+a_1-(a_2-a_1)
\end{align}

\end{document}
