\documentclass[11pt]{article}
\usepackage[a4paper,margin=1.5cm,landscape]{geometry}
\usepackage{booktabs}
\usepackage{array}
\usepackage{multicol}
\usepackage{amsmath,amssymb}
\usepackage{helvet}
\renewcommand{\familydefault}{\sfdefault}
\pagenumbering{gobble}

\begin{document}
\begin{center}
  {\LARGE \textbf{Hoderegningsark – R2}}\\[2mm]
  {\large Gangetabell, kvadrattall og kvadratrøtter}
\end{center}
\vspace{4mm}

% =========================
% 1) Gangetabell 1–12
% =========================
\section*{Gangetabell $1\times 1$ til $12\times 12$}
\setlength{\tabcolsep}{6pt}
\renewcommand{\arraystretch}{1.15}
\small
\begin{center}
\begin{tabular}{c|*{12}{r}}
 & 1 & 2 & 3 & 4 & 5 & 6 & 7 & 8 & 9 & 10 & 11 & 12 \\
\hline
1  & 1  & 2  & 3  & 4  & 5  & 6  & 7  & 8  & 9  & 10  & 11  & 12  \\
2  & 2  & 4  & 6  & 8  & 10 & 12 & 14 & 16 & 18 & 20  & 22  & 24  \\
3  & 3  & 6  & 9  & 12 & 15 & 18 & 21 & 24 & 27 & 30  & 33  & 36  \\
4  & 4  & 8  & 12 & 16 & 20 & 24 & 28 & 32 & 36 & 40  & 44  & 48  \\
5  & 5  & 10 & 15 & 20 & 25 & 30 & 35 & 40 & 45 & 50  & 55  & 60  \\
6  & 6  & 12 & 18 & 24 & 30 & 36 & 42 & 48 & 54 & 60  & 66  & 72  \\
7  & 7  & 14 & 21 & 28 & 35 & 42 & 49 & 56 & 63 & 70  & 77  & 84  \\
8  & 8  & 16 & 24 & 32 & 40 & 48 & 56 & 64 & 72 & 80  & 88  & 96  \\
9  & 9  & 18 & 27 & 36 & 45 & 54 & 63 & 72 & 81 & 90  & 99  & 108 \\
10 & 10 & 20 & 30 & 40 & 50 & 60 & 70 & 80 & 90 & 100 & 110 & 120 \\
11 & 11 & 22 & 33 & 44 & 55 & 66 & 77 & 88 & 99 & 110 & 121 & 132 \\
12 & 12 & 24 & 36 & 48 & 60 & 72 & 84 & 96 & 108& 120 & 132 & 144 \\
\end{tabular}
\end{center}

\vspace{3mm}
\noindent\textit{Tips:} 9-gangen har tverrsum 9 (f.eks. $9\cdot7=63$), tall ganger 5 ender på 0 eller 5, og \(a\cdot b=b\cdot a\) (symmetri i tabellen).

% =========================
% 2) Kvadrattall
% =========================
\section*{Kvadrattall}
\begin{multicols}{3}
\small
\begin{tabular}{@{}lr@{}}
$1^2$  &= 1   \\
$2^2$  &= 4   \\
$3^2$  &= 9   \\
$4^2$  &= 16  \\
$5^2$  &= 25  \\
$6^2$  &= 36  \\
$7^2$  &= 49  \\
$8^2$  &= 64  \\
$9^2$  &= 81  \\
$10^2$ &= 100 \\
\end{tabular}

\columnbreak
\begin{tabular}{@{}lr@{}}
$11^2$ &= 121 \\
$12^2$ &= 144 \\
$13^2$ &= 169 \\
$14^2$ &= 196 \\
$15^2$ &= 225 \\
$16^2$ &= 256 \\
$17^2$ &= 289 \\
$18^2$ &= 324 \\
$19^2$ &= 361 \\
$20^2$ &= 400 \\
\end{tabular}

\columnbreak
\begin{tabular}{@{}lr@{}}
$21^2$ &= 441 \\
$22^2$ &= 484 \\
$23^2$ &= 529 \\
$24^2$ &= 576 \\
$25^2$ &= 625 \\
$26^2$ &= 676 \\
$27^2$ &= 729 \\
$28^2$ &= 784 \\
$29^2$ &= 841 \\
$30^2$ &= 900 \\
\end{tabular}
\end{multicols}

% =========================
% 3) Vanlige kvadratr\o tter
% =========================
\section*{Vanlige kvadratr\o tter (eksakte og tiln\ae rminger)}
\small
\begin{center}
\begin{tabular}{lrr@{\quad}lrr}
\toprule
\textbf{Rot} & \textbf{Eksakt} & \textbf{Ca.} & \textbf{Rot} & \textbf{Eksakt} & \textbf{Ca.} \\
\midrule
$\sqrt{2}$  & $\sqrt{2}$  & 1.414 & $\sqrt{8}$  & $2\sqrt{2}$ & 2.828 \\
$\sqrt{3}$  & $\sqrt{3}$  & 1.732 & $\sqrt{10}$ & $\sqrt{10}$ & 3.162 \\
$\sqrt{5}$  & $\sqrt{5}$  & 2.236 & $\sqrt{12}$ & $2\sqrt{3}$ & 3.464 \\
$\sqrt{6}$  & $\sqrt{6}$  & 2.449 & $\sqrt{15}$ & $\sqrt{15}$ & 3.873 \\
$\sqrt{7}$  & $\sqrt{7}$  & 2.646 & $\sqrt{16}$ & $4$          & 4.000 \\
\bottomrule
\end{tabular}
\end{center}

\vspace{2mm}
\noindent\textit{Husk forenkling:} $\sqrt{ab}=\sqrt{a}\,\sqrt{b}$ for $a,b\ge0$ (f.eks. $\sqrt{12}=\sqrt{4\cdot3}=2\sqrt{3}$). Og $\,(a\pm b)^2=a^2\pm2ab+b^2\,.$

\vspace{2mm}
\noindent\textbf{Ekstra mentaltriks:} $(n\pm1)^2=n^2\pm2n+1$ (lett å gå fra $n^2$ til neste kvadrat), og $\sqrt{n^2\cdot m}=n\sqrt{m}$.

\end{document}
