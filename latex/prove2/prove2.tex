\documentclass{article}

% Language setting
% Replace `english' with e.g. `spanish' to change the document language
\usepackage[english]{babel}

% Set page size and margins
% Replace `letterpaper' with `a4paper' for UK/EU standard size
\usepackage[letterpaper,top=2cm,bottom=2cm,left=3cm,right=3cm,marginparwidth=1.75cm]{geometry}

% Useful packages
\usepackage{amsmath}
\usepackage{amssymb}
\usepackage{graphicx}
\usepackage{inconsolata}
\usepackage{minted}
\usepackage[colorlinks=true, allcolors=blue]{hyperref}

\title{R2 Eksamen}
\author{Skrevet av André Hansen}

\begin{document}
\maketitle

\begin{abstract}
Dette er en template til selve eksamen
\end{abstract}

\section{Oppgave 1}

\subsection{a)}

Først modelerer jeg ballen i Geogebra

Høyden kan jeg finne ved å se se $z$ aksen av $\vec{r}(0)$

Og posisjonen etter $0.5$ kan jeg finne ved å løse $\vec{r}(0.5)$

\begin{figure}[H]
    \centering
    \includegraphics[width=0.7\textwidth]{bilder/oppgave1.a.png}
    \caption{Utregning i CAS}
\end{figure}

Høyden over kanten er $6m$

Posisjonen etter $0.5s$ er $(1, 2, 4.46)$

\subsection{b)}

For å finne farten når ballen treffer bakken må jeg finne lengen av den deriverte sekundet z posisjonen er 0

Jeg kan finne sekundet ballen treffer bakken ved å løse ligningen $z(\vec{r}(t))=0, \quad t > 0$

Derreter kan jeg finne farten å bruke løsiningen $l$ ved : $|\vec{r}'(l)|$

\begin{figure}[H]
    \centering
    \includegraphics[width=0.7\textwidth]{bilder/oppgave1.b.png}
    \caption{Utregning i CAS}
\end{figure}

Farten til ballen når den trefer bakken er ca$10.22m/s$

\subsection{c)}

For å finne sekundet farten er $10m/s$ kan vi gjøre det ved å sette lengden til den deriverte lik $10m/s$:

$|r'(t)|=10$

\begin{figure}
    \centering
    \includegraphics[width=0.7\textwidth]{bilder/oppgave1.c.png}
    \caption{Utregning i CAS}
\end{figure}

Etter ca $8.4 s$ er farten til ballen $10m/s$

\section{Oppgave 2}

\subsection{a)}

Påstanden om at 3 punkter i et plan kan bestemme likningen til planet stemmer.

Dette stemmer fordi at normalvektoren til et plan står altid vinkerett på alle vektorer i planet.

Dermed vil kryssproduktet av to vektorer som dannes av tre punkter i planet danne en normalvektor til planet.

Videre vet vi også et punkt i planet.

Med normalvektor og et punkt i planet definert har vi alle komponententer for likningen til planet.

\subsection{b)}

Vi har: $S(x)=1+(ln x - 1) + (ln x - 1)^2 + ...$

Den kan skrives om slik : $S(x)=\sum_{i=0}^{\infty} (ln x - 1)^i$

Vi kan bruke det vi vet om summen av en uendelig geometrisk rekke for å skrive rekken om til en eksplisitt formel

\begin{align}
    a \sum_{i=0}^{\infty} r^i &= a \frac{1}{1-r}, \quad |r| < 1 \\
    S(x) &= \sum_{i=0}^{\infty} (ln x - 1)^i = \frac{1}{1-ln x - 1} = \frac{1}{ln x} 
    \text{setter in $\frac{1}{e}$ for å teste}
    \frac{1}{ln \frac{1}{e}} = -1
\end{align}

Dermed stemmer påstanden ikke

\subsection{c)}

Vi kan teste påstanden med å sammenligne integralene

\begin{align*}
    \int f(x) \equiv \int g(x), \quad a \in \mathbb{R}, a > -1 \\
\end{align*}

For å teste påstanden, testet jeg med $a=1$

\begin{figure}
    \centering
    \includegraphics[width=0.7\textwidth]{bilder/oppgave2.c.png}
    \caption{Utregning i CAS}
\end{figure}

Dermed stemmer påstanden ikke

\section{Oppgave 3}

Jeg løser denne oppgaven med regresjonsanalyse.

Etter å ha modulert en halv sideflate kan jeg regne volumet å fine volumet til omdreingslegemet>

$$V=\pi \int_{0}^{3.87} f(x)^2$$

Hvor $f$ er funksjon for jorbæret og $3.87$ er bunkunktet

Antar at enhetene er cm.

jorbæret har en volum på ca $171.17 cm^3$

\begin{figure}[H]
    \centering
    \includegraphics[width=0.7\textwidth]{bilder/oppgave3.png}
    \caption{Utregning i CAS}
\end{figure}

\section{Oppgave 4}

\subsection{a)}

Vi har funksjonen for fart gitt ved $v(t)=-6 \sin(360t - \frac{\pi}{2}) + 54$

Dermed kan vi finne gjennomsnittsfargen for et tidspunkt med (v(0)-v(x))/2

\subsection{b)}

Vi kan finne ut av når bilen har størst akselerasjon ved å løse den deriverte av akselerasjonen lik 0 og sjekke for konkavitet
 
$v''(t)=0$

\begin{figure}[H]
    \centering
    \includegraphics[width=0.7\textwidth]{bilder/oppgave4.b.png}
    \caption{Utregning i CAS}
\end{figure}

Den største akselerasjonen er på $  1036 m/s^2$

\subsection{c)}

Vi man finne hvor lang tid det tar før de har kjørt $2km$ å sjekke når distande er lik 2. Vi kan finne distanse ved å integrere farten

Formelen blir da $\int v(t)dt=2$ Hvor $C=0$

\begin{figure}[H]
    \centering
    \includegraphics[width=0.7\textwidth]{bilder/oppgave4.c.png}
    \caption{Utregning i CAS}
\end{figure}

Det tar de ca $2.2 min$ å kjøre $2.0 km$

\section{Oppgave 5}

\section{Oppgave 6}

\bibliographystyle{alpha}
\bibliography{sample}

\end{document}
