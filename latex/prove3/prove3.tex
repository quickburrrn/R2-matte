\documentclass{article}

% Language setting
% Replace `english' with e.g. `spanish' to change the document language
\usepackage[english]{babel}

% Set page size and margins
% Replace `letterpaper' with `a4paper' for UK/EU standard size
\usepackage[letterpaper,top=2cm,bottom=2cm,left=3cm,right=3cm,marginparwidth=1.75cm]{geometry}

% Useful packages
\usepackage{amsmath}
\usepackage{amssymb}
\usepackage{graphicx}
\usepackage{inconsolata}
\usepackage{minted}
\usepackage[colorlinks=true, allcolors=blue]{hyperref}

\title{R2 Prøve 3}
\author{Skrevet av André Hansen}

\begin{document}
\maketitle

\begin{abstract}
Dette er en template til selve eksamen
\end{abstract}

\section{Oppgave 1}

\subsection{a)}

Jeg fant høyden til bilen ved å se på z kordinaten til bilen etter 5 sekunder

$$Høyde = z(r(5)) = \frac{20}{3}$$

\begin{figure}[H]
    \centering
    \includegraphics[width=0.7\textwidth]{bilder/oppgave1a1.png}
    \caption{Utregning i CAS}
\end{figure}

\subsection{b)}

Fartsverkoren kan finnes ved å definere en funksjon for fart $v(t)=r'(t)$, derreter se vektoren etter $10s$

\begin{figure}[H]
    \centering
    \includegraphics[width=0.7\textwidth]{bilder/oppgave1b1.png}
    \caption{Utregning i CAS}
\end{figure}

\subsection{}{c)}

For å finne høyden til hver etasje antar jeg at det en utkjørsel for biler per periode. Jeg antar også at enhetene er gitt i meter.

Dermed finner jeg en mulig avstand mellom etasjene med å regne avstanden mellom z aksene på kurven

\section{Oppgave 2}

\subsection{a)}

For å finne ut av hvor mye Nora må sette inn årlig løser vi en sum likning med $n=29$ fordi det er totalt 29 år med sparing

$$a \cdot \sum_{i=1}^{29} 0.025^i=3750000$$

\subsection{b)}

\begin{figure}[H]
    \centering
    \includegraphics[width=0.7\textwidth]{bilder/oppgave2b1.png}
    \caption{Utregning i CAS}
\end{figure}

\section{Oppgave 3}

\subsection{a)}

Tiden det tar før akselerasjonen til haren er null er ca 1 sekund.

Jeg fand dette svaret ved å definere en funksjon for akselerasjon $a(t)=v'(t)$

og løse likningen $a(t)=0$

Dette forteller oss hvor lang tid haren går fra stillestående til dens toppfart.

\begin{figure}[H]
    \centering
    \includegraphics[width=0.7\textwidth]{bilder/oppgave3a1.png}
    \caption{Utregning i CAS}
\end{figure}

\subsection{b)}

Haren løper ca 103 meter.

Jeg fant dette ved å regne ut

$$\int_{0}^{7} v(t)$$

I CAS

\begin{figure}[H]
    \centering
    \includegraphics[width=0.7\textwidth]{bilder/oppgave3b1.png}
    \caption{Utregning i CAS}
\end{figure}

\subsection{c)}

For å finne gjennomsnitsfarten de første 200 meterene løste jeg først:

$$\int_{0}^{x} v(x) = 200 dx$$

gjennomsnitsfarten de første 200 meterne

Derreter regner jeg ut

$$\int_{1}^{tid} v(t) dt$$

Gjennomsnitsfarten er ca $13.4$ meter

\begin{figure}[H]
    \centering
    \includegraphics[width=0.7\textwidth]{bilder/oppgave3c1.png}
    \caption{Utregning i CAS}
\end{figure}

\bibliographystyle{alpha}
\bibliography{sample}

\end{document}
